%Made By Thomas Debelle
%Ajouté des Packages si nécessaires
\documentclass{exam}
\usepackage[a4paper, total={6in, 9in}]{geometry}
\usepackage[utf8]{inputenc}
\usepackage[french]{babel}
\usepackage{graphicx}
\usepackage[T1]{fontenc}
\usepackage{amsmath}
\usepackage{hyperref}
\usepackage{amssymb}
\usepackage{listings}
\usepackage{xcolor}
\usepackage{array}
\usepackage{float}
\usepackage{amsfonts}
\usepackage{titlesec}
\usepackage{xparse}
\usepackage{framed}
\usepackage{wrapfig}

\definecolor{codegreen}{rgb}{0,0.6,0}
\definecolor{codegray}{rgb}{0.5,0.5,0.5}
\definecolor{codepurple}{rgb}{0.58,0,0.82}
\definecolor{backcolour}{rgb}{1.0,1.0,1.0}
\definecolor{codeblue}{rgb}{0,0,0.8}

\lstdefinestyle{mystyle}{
    backgroundcolor=\color{backcolour},   
    commentstyle=\color{codegray},
    keywordstyle=\color{codeblue},
    numberstyle=\tiny\color{codegray},
    stringstyle=\color{codeblue},
    basicstyle=\ttfamily\footnotesize,
    breakatwhitespace=false,         
    breaklines=true,                 
    captionpos=b,                    
    keepspaces=true,                 
    numbers=left,                    
    numbersep=5pt,                  
    showspaces=false,                
    showstringspaces=false,
    showtabs=false,                  
    tabsize=2,
    frame=shadowbox
}
\lstset{style=mystyle}
\lstset{language=Oz}

\hypersetup{
    colorlinks=true,
    linkcolor=black,
    filecolor=magenta,
    urlcolor=cyan,
    pdftitle={Overleaf Example},
    pdfpagemode=FullScreen,
    }
\begin{document}
\begin{center}
\fbox{\fbox{\parbox{5.5in}{\centering
Solution proposée pour l'examen Septembre 2022 de LINFO1104}}}
\end{center}
Solutions par @norahrmoire et ré-écris en Latex par @tfloow sur discord. Si vous trouvez une faute merci de le signaler \href{https://github.com/Tfloow/Q4_EPL/pulls}{ici}.


\begin{questions}

\question Translate the following program into kernel language. Be careful to do a complete translation that uses only kernel language instructions!: \begin{lstlisting}[escapechar=\%]
local F C in
  C={NewCell 1}
  fun {F A}
    @A
  end
  C:={F C}+1
end
\end{lstlisting}

\question Basic concepts.
\begin{parts}
\part Define "scope" in a programming language. Give an example.

\part Define "contextual environment" of a procedure. Give an example.
\end{parts}
\end{questions}



\end{document}
