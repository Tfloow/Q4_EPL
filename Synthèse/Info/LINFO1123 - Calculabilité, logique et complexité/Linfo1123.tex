%Made By Thomas Debelle
%Ajouté des Packages si nécessaires
\documentclass{report}
\usepackage[a4paper, total={6in, 9in}]{geometry}
\usepackage[utf8]{inputenc}
\usepackage[francais]{babel}
\usepackage{graphicx}
\usepackage{graphics}
\usepackage[T1]{fontenc}
\usepackage{amsmath}
\usepackage{hyperref}
\usepackage{amssymb}
\usepackage{listings}
\usepackage{xcolor}
\usepackage{array}
\usepackage{float}
\usepackage{amsfonts}
\usepackage{fancyhdr}
\usepackage{titlesec}
\usepackage{xparse}
\usepackage{wrapfig}

\hypersetup{
    colorlinks=true,
    linkcolor=black,
    filecolor=magenta,
    urlcolor=cyan,
    pdftitle={Overleaf Example},
    pdfpagemode=FullScreen,
    }
\begin{document}

%Si la mention "Juin 2023" est sur une autre page, changé le dernier VSPACE
\begin{titlepage}
    \begin{figure}
        \includegraphics[height = 2cm]{UCL_Logo.png}
        \label{fig:my_label}
    \end{figure}

    \hspace*{100cm}
    \centering
    \vspace*{7cm}

    {\Huge \textbf{Résumé de LINFO1123}}\\
    \vspace*{0.25cm}
    compilation du \today\\
    \vspace*{0.25cm}
    \Large{Thomas Debelle}\\

    \vspace*{9.5cm} %Le dernier VSPACE
    {\Large Juin 2023}
\end{titlepage}

%_____NE PAS MODIFIER______
\tableofcontents
\newpage

\section*{Préface}

Bonjour à toi !\\

Cette synthèse recueille toutes les informations importantes données au cours, pendant les séances de tp et est amélioré grâce au note du Syllabus. Elle ne remplace pas le cours donc écoutez bien les conseils et potentielles astuces que les professeurs peuvent vous donner. Notre synthèse est plus une aide qui on l'espère vous sera à toutes et tous utiles.\\

Elle a été réalisée par toutes les personnes que tu vois mentionné. Si jamais cette synthèse a une faute, manque de précision, typo ou n'est pas à jour par rapport à la matière actuelle ou bien que tu veux simplement contribuer en y apportant ta connaissance ? Rien de plus simple ! Améliore la en te rendant \href{http://www.github.com/Tfloow/Q4_EPL}{ici} où tu trouveras toutes les infos pour mettre ce document à jour. (\textit{en plus tu auras ton nom en gros ici et sur la page du github})\\

Nous espérons que cette synthèse te sera utile d'une quelconque manière ! Bonne lecture et bonne étude.

%_______Vous pouvez modifier en-dessous_____
\chapter{Concepts}
Dans ce chapitre, on s'intéresse aux ensembles, cardinalité et équipotences de ces derniers
\section{Ensemble}
Un ensemble est un \textit{collection} d'objets, \textit{sans répétition}, ces derniers sont appelés \textit{éléments} de l'ensemble.
Donc un ensemble peut être des chiffres, des lettres, il peut être vide symbolisé par $void$ %à ajouter
On peut réaliser des opérations dessus, on peut déterminer des \textit{sous-ensembles d'ensemble} donc des ensembles issus d'ensemble.
On a également une notion s'appelant le \textit{complément} d'un ensemble dénoté $\tilde{A}$\\%trouver le bon charactère

\subsubsection{Langage}
Un \textit{langage} n'est autre qu'un mot ou bien un ensemble de caractères d'une taille fixée. Une chaine vide est écrite via le caractère "$\epsilon$".\\
On forme un langage via un \textit{alphabet} qui n'est autre qu'un ensemble de symboles, on le dénote "$\Sigma$". Tout langage est donc une suite de symbole issue de l'\textit{alphabet}.\\
$\Sigma^*$ correspond à l'ensemble des langages formés via l'alphabet.

\subsubsection{Relations}
Lorsque nous avons deux ensembles appelés \textit{A} et \textit{B}, on peut établir une relation appelé \textcolor{brown}{R} qui nous donne un sous-ensemble $AXB$ %trouver le symbole pour cross product
On peut représenter la relation par une table.

\subsubsection{Fonctions}
Lorsque nous avons deux ensembles appelés \textit{A} et \textit{B}, on peut avoir ce qu'on appelle une \textit{fonction} \textcolor{brown}{F}. C'est une relation tel que:
\begin{equation}
\exists a \in A : \exists b \in B :\quad <a,b> \quad \in f
\end{equation}
Il n'existe pas plus d'un b pour un a. Si pour un a il n'existe pas de b, on dit que $f(a)$ est indéfini et donc $f(a) = \perp$ ou \textit{bottom}.

\subsubsection{Propriétés des fonctions}
\begin{itemize}
\item un \textcolor{brown}{domaine de fonction} ou dom(f) $= {a \in A | f(a) \neq \perp}$
\item une \textcolor{brown}{image de fonciton} ou image(f) $= b \in B | \exists a \in A: b = f(a)$
\item f est dit \textcolor{brown}{fonction totale} si dom(f) $= A$
\item f est dit \textcolor{brown}{fonction partielle} si dom(f) $\in A$ % trouver le symbole pour include
\item f est \textcolor{brown}{surjectif} ssi image(f) = B autrement dit, tout élement est associé à minimum 1 élément dans B.
\item f est \textcolor{brown}{injectif} ssi $\forall a, a' \in A: a \neq a' \Rightarrow f(a)\neq f(a')$ autrement dit on ne fait correspondre qu'au plus un élément de A dans B.
\item f est \textcolor{brown}{bijectif} s'il combine \textit{surjectif} et \textit{injectif}
\end{itemize}

Intéressons nous aux \textcolor{red}{extensions} qui est le fait de rajouter une fonction qui ne définit un élément de B pas encore défini.
\begin{equation}
\forall x \in A : g(x) \neq \perp \Rightarrow f(x) = g(x)
\end{equation}	
f à la même valeur que g partout où g est défini.

\subsubsection{Définition d'une fonction}
Comme dit précédemment, une fonction est défini par sa table. On va souvent utiliser une description de la table qui permet que celle-ci soit clair et bien défini. De plus, on a pas besoin de savoir comment calculer ceci.\\
On peut également définir une table via une fonction ou un algorithme.

\section{Ensemble énumérable}
On dit que 2 ensembles ont le même cardinal (\textit{A} et \textit{B}) ssi il existe une bijection entre ces 2 ensembles. Donc chaque élément de \textit{A} correspond à un élément de \textit{B}.\\

On dit d'un ensemble qu'il est dénombrable ssi il est \textcolor{red}{fini} ou il existe une \textcolor{red}{bijection} entre l'ensemble $\mathbb{N}$ et cet ensemble.

\subsubsection{Exemples}
\begin{itemize}
\item L'ensemble $\mathbb{Z}$
\item L'ensemble des nombres pairs
\item Des paires d'entiers
\item L'ensemble des programmes Java
\end{itemize}

\subsubsection{Propriétés}
Tout sous-ensemble d'ensemble énumérable est \textit{énumérable}. L'union et l'intersection d'ensembles énumérables est \textit{énumérable}.\\

En s'intéressant à l'ensemble des programmes informatiques, on se rend compte que c'est une \textit{ensemble énumérable infini}. De plus, les programmes informatiques ne considèrent que des choses \textit{énumérables}.

\section{Cantor}
Le théorème de \textit{Cantor} nous dit que l'ensemble des nombres entre 0 et 1 compris est \textit{non énumérable}.
\begin{equation}
E = {x \in \mathbb{R} | 0 < x \le 1}
\end{equation}

\subsubsection{Preuve}
Pour prouver cela, on va réaliser une table et on va réaliser une \textit{diagonalisation de Cantor}.
\begin{center}
\begin{tabular}{|c||c|c|c|c|c|}
	\hline
	 & chiffre 1 & chiffre 2 & ... & chiffre $k+1$ & ...\\
	 \hline
	 $x_0$ & \textcolor{red}{$x_{00}$} & $x_{01}$ & ... & $x_{0k}$ & ...\\
	 \hline
	 $x_1$ & $x_{10}$ & \textcolor{red}{$x_{11}$} & ... & $x_{1k}$ & ...\\
	 \hline
	 ...& ... & ... & ... & ... & ...\\
	 \hline
	 $x_k$ & $x_{k0}$ & $x_{k1}$ & ... & \textcolor{red}{$x_{kk}$} & ...\\
	\hline
	 ...& ... & ... & ... & ... & ...\\
	\hline
\end{tabular}
\end{center}
Ensuite, on va définir notre nombre de la diagonale qui vaut $d = 0.x_{00}x_{11}...x_{kk}$. De cet valeur, on va créer une valeur $d'$ qui a comme propriété $x_{kk} \neq x'_{kk} \forall k$.\\
Mais, on doit stocker notre valeur $d'$ dans la table. On la stock à $p$ ce qui donne $d' = 0.x'_{p0}x'_{p1}... \textcolor{red}{x'_{pp}}$ mais à cause de la construction de $d = 0.x_{00}x_{11}...\textcolor{red}{x_{pp}}$. Par construction, $x'_{pp} \neq x_{pp}$ mais cela ne peut être respecté. Donc, \textbf{il n'y a pas} de \textit{bijection} des $\mathbb{N}$ vers cet ensemble. Donc cet ensemble est \textit{non énumérable}.

\subsubsection{Autre ensemble non énumérable}
\begin{itemize}
\item L'ensemble des $\mathbb{R}$.
\item L'ensemble des sous-ensemble de $\mathbb{N}$.
\item L'ensemble des chaines infinies de caractères d'un alphabet fini.
\item L'ensemble des \textit{fonctions} de $\mathbb{N}$ dans $\mathbb{N}$.
\end{itemize}
Chose intéressante à noter, comme on a une infinité non énumérable de fonctions $\mathbb{N}$ dans $\mathbb{N}$ et un nombre de programme informatique \textit{infini énumérable}. On ne peut résoudre tous les problèmes informatiques donc.

\end{document}
